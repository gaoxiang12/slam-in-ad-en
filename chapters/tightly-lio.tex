% !Mode:: "TeX:UTF-8"
\thispagestyle{empty}
\chapter{Tightly Coupled LIO System}
\label{cpt:tightly-lio}
\thispagestyle{empty}

From this chapter through the next three chapters, we will introduce readers to some practical autonomous driving localization and mapping technologies. This chapter focuses on the tightly coupled Lidar-IMU-Odometry system, known as the tightly coupled LIO system (sometimes referred to as Lins\cite{Qin2020} in literature, meaning a system combining LiDAR and inertial navigation). Chapters~\ref{cpt:mapping} and~\ref{cpt:localization} will cover offline mapping systems and real-time fusion localization techniques respectively. Overall, autonomous driving applications of SLAM primarily concentrate on two major modules: mapping and localization. Within the localization module, DR or LIO is often used for local position estimation. The tightly coupled LIO system is more complex than the loosely coupled LIO system introduced in the previous chapter. This chapter will first explain its principles before proceeding with implementation.

\includepdf[width=\textwidth]{art/ch8.pdf}

\section{Principles and Advantages of Tight Coupling}
First, let's address a fundamental question: What is tight coupling, and why do we need a tightly coupled LIO system? The loosely coupled LIO introduced in the previous chapter already demonstrates good performance - what additional benefits can tight coupling provide? In fact, the term ``\textbf{tightly coupled}'' isn't unique to LIO systems; similar concepts exist in traditional integrated navigation and VIO fields\cite{yang2019tightly,liu2018implementation,kong2015tightly}. Broadly speaking, any state estimation system that considers the intrinsic properties of sensors rather than \textbf{modularly} fusing their outputs can be called a tightly coupled system\cite{Soloviev2008}. For example, systems considering IMU observation noise and biases can be called tightly coupled IMU (or INS) systems; those accounting for LiDAR registration residuals are tightly coupled LiDAR systems; while systems incorporating visual feature reprojection errors or RTK sub-states and satellite counts qualify as tightly coupled visual or RTK systems\cite{Shi2012,Schleicher2009}. In loosely coupled systems, we can treat each sensor or algorithm module as a black box, considering only their outputs. The previous chapter demonstrated this approach, which readers should now understand.

So what specific advantages does tight coupling offer? Based on the author's experience, when all algorithm modules function normally, there may be no significant difference between tightly and loosely coupled systems. For instance, when fusing GINS systems with LiDAR odometry, if RTK signals remain valid, the coupling approach shouldn't produce noticeable differences. However, in real-world systems, individual algorithm modules often can't maintain continuous normal operation. Standalone IMU systems quickly diverge without velocity and position observations; independent LiDAR and visual odometry may fail or degrade in structurally poor environments. In loosely coupled systems, when a module fails, we must logically identify the failure and attempt recovery. Tightly coupled systems allow one module's operational status to directly influence others, helping better constrain their working dimensions. Taking loosely coupled LIO as an example: when a vehicle traverses degenerate areas, LiDAR odometry relying solely on point cloud matching may fail, producing erroneous pose estimates with extra degrees of freedom that could mislead the entire system after fusion. In tightly coupled LIO systems, states remain constrained by other sensors, keeping these degrees of freedom within fixed bounds to maintain system validity. While this explanation may seem abstract, this chapter will use practical cases to help readers better understand these concepts.